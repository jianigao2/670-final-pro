\documentclass[12pt]{article}
%[10pt,technote]{IEEEtran}
\usepackage{hyperref}
\usepackage{graphicx}
\usepackage[affil-it]{authblk}
\usepackage{color}
\usepackage{amsgen,amsmath,amstext,amsbsy,amsopn,amssymb}
\usepackage{geometry}
\usepackage{subcaption}
\usepackage{caption}
\usepackage{wrapfig}
\usepackage{comment}
\usepackage{mathrsfs}
\usepackage{upgreek}
\usepackage{amssymb}
\usepackage{textcomp}
\usepackage{amsmath}
\usepackage{tcolorbox}
\usepackage{listings}
\usepackage{fancyhdr}
\usepackage{cite}
\usepackage{wasysym} 
%\usepackage{subfigure}
%\usepackage{wraptable}

\geometry{left=2.5cm, right=2.5cm,top=2.5cm,bottom=2.5cm}
\makeatletter
\newcommand{\rmnum}[1]{\romannumeral #1}
\newcommand{\Rmnum}[1]{\expandafter\@slowromancap\romannumeral #1@}
\makeatother


\setcounter{secnumdepth}{0}
\title{\bf Principle 4 }
\author{Jiani Gao\\Elisa Taveras Pena\\Iryna Anufriyeva\\ Thanh Pham\\ Justin Kizner}
\affil{Department of Economics, Binghamton University}
\date{\today}
\pagestyle{fancy}
\chead{} \lhead{}\rhead{}
\cfoot{jgao30@binghamton.edu}

\begin{document}
\maketitle
\newpage
%\tableofcontents	

\newpage
\noindent

\section{To implement Principle 4, we need:}
\begin{itemize}
\item[\Rmnum{1}.] 
 \textbf{\textcolor{red}{Two types of agents}: fisherman and monitor}
 \item[\Rmnum{2}.]
\textbf{\textcolor{red}{Emergence rule}}
\end{itemize}
 
\textit{\textcolor{blue}{Red words are specifically relevant to Principle 4.}}
\begin{enumerate}
	
	\item Fisherman
	\begin{itemize}
		\item Fishermen take fish from lake.
		\item There are 24 hours per day, people go fish in the first 10 hours, and then go home to take rest.
		\item Fishermen get happiness from taking fish, and different types of fish will give them different happiness. Specifically, carp: 1 unit of happiness, salmon: 2 units of happiness, tilapia: 6 units of happiness.
		\item Cost of fishing: lose 0.1 happiness for every step they move.
		\item Chance of catching a fish per tick: similar to wolves-sheep model, people wonder around and find preys.
		\item New people showing up? leaving?: I let people leave if one is unhappy 5 days in a row.People move in if the percentage of happy fishermen is greater than 80\%.
		\item \textcolor{red}{Each fisherman has a probability to break rules at the start of each day. If one is unhappy, the probablity one breaks rules will increase.}
		\item \textcolor{red}{Rule breakers catch more fish than daily quota.}
		\item \textcolor{red}{Rule breakers who are caught for more than 3 times, they leave. For the first times, they just give away half of their happiness. For the second time, they give away 75\% happiness (graduated sanction).}
	\end{itemize}

    
    \item Monitor
    \begin{itemize}
    	\item \textcolor{red}{Monitors check fishermen within a certain distance every night.}
    	\item \textcolor{red}{Monitor is set to be happy, so they will not move out. He take away 50\% happiness from rule breakers.}
    	\item {\textcolor{red}{If monitor catches zero rule breaker, he will trun into a fisherman.}}
    \end{itemize}

    \item Emergence
    \begin{itemize}
    	\item \textcolor{red}{At the beginning, we have one monitor.}
    	\item \textcolor{red}{Community decide to increase the number of monitors if one of the two cases below happen:}
    	\[	\left\{
    		\begin{array}{cc}
    		A.&\text{The number of fish in the lake is less than 50\% of the original lebel}\\
    		B.&\text{The number of unhappy fishermen is greater than 1/3}	
    		\end{array}
    		\right    	\]
    	\item \textcolor{red}{New monitor is selected randomly from fishermen who are unhappy. (collective decision)}	
    	\item \textcolor{red}{\textit{System trys to reach the steady state where the above 2 conditions are not violated with the right number of monitors.}}		
    \end{itemize}
    
  
    \item Fish
    \begin{itemize}
    	\item There are 3 fishes: carp, salmon, tilapia,
    	\item They reproduce at different rates.
    	\item Different types of fish have different size and values.
    \end{itemize}

    
    \item Environment
    \begin{itemize}
    	\item One lake.
    	\item \textcolor{red}{Link patches outside the lakes with agents, so everybody has a fixed location.}
    \end{itemize}


    \item Programming logic:
    \begin{itemize}
    	\item Program physical environment first? \textbf{Done this part.} 
    	\item Then: the "social" environment of rules? \textbf{In progress.} 
    \end{itemize}
  
    
     
\end{enumerate}

\noindent
\textbf{
We will build a model without monitors first, and see the average level of happiness. Then we will introduce monitors into the model, and find out the average level of happiness. By comparing these two results, we can know which one is more successful. We will use behavior space to find out the optimal level of monitors.	}




%\bibliographystyle{unsrt} 
%\bibliography{}
\end{document}
